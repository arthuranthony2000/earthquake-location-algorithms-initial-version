%Definindo nomes de acordo com o idioma usado no documento
\newcommand\resumoname{Resumo} 
\newcommand\abstractname{Abstract} 
\newcommand\agradecimentosname{Agradecimentos}
\newcommand\epigrafename{Epígrafe}
\newcommand\simbolosname{Lista de Símbolos}
\newcommand\abreviaturasname{Lista de Abreviaturas}

\if\boolean{PPgSCIngles}{
\renewcommand\agradecimentosname{Acknowledgements}
\renewcommand\epigrafename{Quote}
\renewcommand\simbolosname{List of Symbols}
\renewcommand\abreviaturasname{List of Acronyms}
}

%Definindo environment abstract para classe scrbook (não faz parte da classe original)
\makeatletter
\if@titlepage
\newenvironment{abstract}[1][\abstractname]{%
	\titlepage
	\null\vfil
	\@beginparpenalty\@lowpenalty
	\begin{center}%
		\bfseries \abstractname
		\@endparpenalty\@M
    \end{center}}%
{\par\vfil\null\endtitlepage}
%\else
%\newenvironment{abstract}{%
%	\if@twocolumn
%	\section*{\abstractname}%
%	\else
%	\small
%	\begin{center}%
%		{\bfseries \abstractname\vspace{-.5em}\vspace{\z@}}%
%	\end{center}%
%	\quotation
%	\fi}
%{\if@twocolumn\else\endquotation\fi}
%\fi
\makeatother

%Definindo environment agradecimentos
\newenvironment{agradecimentos}[1][\agradecimentosname]{%
	\begin{center}
		{\Large \bfseries \agradecimentosname}
	\end{center}
}


%Definindo environment epígrafe
\newenvironment{epigrafe}[1][\epigrafename]{%
	\begin{center}
		{\Large \bfseries \epigrafename} %Comentar essas 3 linhas se não quiser que o nome Epígrafe ou Quote apareça no cabeçalho
	\end{center}
}

%Definindo environment símbolos
\newenvironment{abreviaturas}[1][\abreviaturasname]{%
	\begin{center}
		{\Large \bfseries \abreviaturasname}
	\end{center}
    \input{editaveis/acronimos.tex}
    \cleardoublepage
}


%\newcommand{\curso}[1]{\def\imprimircurso{#1}}
%\newcommand{\dataDaAprovacao}[1]{\def\imprimirdatadaaprovacao{#1}}
%\newcommand{\tituloEstrangeiro}[1]{\def\imprimirtituloestrangeiro{#1}}


\newcommand{\imprimircapa}{%
  \begin{titlepage}
    \begin{center}

      % Cabeçalho (não deve ser modificado)
      % Contém o brasão da Universidade, o logotipo do Departamento, além dos dados
      % relacionados à vinculação do aluno (Universidade, Centro, Departamento e Curso)
      \begin{minipage}{2cm}
        \begin{center}
          \includegraphics[width=1.8cm, height=2.1cm]{imagens/logos/Brasao-UFRN.jpg}
        \end{center}
      \end{minipage}
      \begin{minipage}{11cm}
        \begin{center}
          \begin{singlespace}
            {\small \textsc{Universidade Federal do Rio Grande do Norte}
\\
                    \textsc{Centro de Ciências Exatas e da Terra}
\\
                    \textsc{Departamento de Informática e Matemática Aplicada}
\\
                    \textsc{Programa de Pós-Graduação em Sistemas e Computação}
\\
                    \ifthenelse{\boolean{PPgSCTese}}
                    {\textsc{Doutorado Acadêmico em Ciência da Computação}}
                    {\textsc{Mestrado Acadêmico em Sistemas e Computação}}
            }
          \end{singlespace}
        \end{center}
      \end{minipage}
      \begin{minipage}{2cm}
        \begin{center}
          \includegraphics[width=1.8cm, height=2cm]{imagens/logos/logo-ppgsc.png}
        \end{center}
      \end{minipage}

      \vspace{5cm}

      % Título do trabalho
      {\setlength{\baselineskip}
      {1.3\baselineskip}
      {\LARGE \textbf{\titulo}}\par}

      \vspace{4cm}

      % Nome do aluno (autor)

      {\large \textbf{\autor}}

      \vspace{6cm}

      % Local da instituição onde o trabalho deve ser apresentado e ano de entrega do mesmo
      \local\\\data 
    \end{center}
  \end{titlepage}

  % Solução para geração de páginas duplicadas, uma delas fica em branco
  \hypersetup{pageanchor=true}
}

% Conteudo padrao da Folha de Rosto - Teste
%\makeatletter

%\newcommand{imprimirfolhaderosto} {
%	\begin{center}
		
%		{\bfseries \Large \autor}
%		\vspace*{\fill}
		
%	\end{center}
%}

%\makeatother

% Conteudo padrao da Folha de Rosto
\makeatletter

\newcommand{\folhaderosto} {
  \begin{center}

    {\bfseries \Large \autor{}}

    \vspace*{\fill}\vspace*{\fill}
    \begin{center}
      \bfseries \Large \titulo
    \end{center}
    \vspace*{\fill}

      \hspace{.45\textwidth}
      \begin{minipage}{.5\textwidth}
        \singlespacing
          \preambulo
      \end{minipage}%
      \vspace*{\fill}

    {\large Orientador~\par\orientador\par}
    \if\boolean{COorientador}{%
      {\large Coorientador \\ \coorientador}
    }%
    \vspace*{\fill}

      \textsc{\instituicao}\vspace*{\fill}

    {\large\local}
    \par
    {\large\data}
    \vspace*{1cm}

  %\end{center}
}

\makeatother

%Novo comando signature, pode ser parametrizado mudando os valores de \signSkip, \signThickness e \signWidth
\newcommand{\signature}{ {\centering{ \vspace{ \signSkip } \parbox[t]{ \signWidth }{\rule[-3pt]{ \linewidth }{ \signThickness } \par\smallskip } } } }
