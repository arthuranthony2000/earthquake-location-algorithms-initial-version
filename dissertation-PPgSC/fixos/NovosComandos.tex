%Definindo nomes de acordo com o idioma usado no documento
\iftoggle{PPgSC-Ingles}{
%\ifthenelse{\boolean{PPgSCIngles}}{
	\newcommand\agradecimentosnome{Acknowledgements}
	\newcommand\epigrafenome{Quote}
	\newcommand\simbolosnome{List of Symbols}
	\newcommand\abreviaturasnome{List of Acronyms}
	\newcommand\algoritmossnome{List of Algorithms}
}
{
	\newcommand\agradecimentosnome{Agradecimentos}
	\newcommand\epigrafenome{Epígrafe}
	\newcommand\simbolosnome{Lista de Símbolos}
	\newcommand\abreviaturasnome{Lista de Abreviaturas}
	\newcommand\algoritmosnome{Lista de Algoritmos}
}

%\renewcommand{\ABNTEXchapterfont}{\scshape \mseries \selectfont}
%\renewcommand{\ABNTEXsectionfont}{\nshape \bfseries \selectfont} 

%Novo comando signature, pode ser parametrizado mudando os valores de \signSkip, \signThickness e \signWidth, que estão definidos no arquivo variaveis.tex
\newcommand{\signature}{  \vspace{ \signSkip } \parbox[t]{ \signWidth }{\rule[-3pt]{ \linewidth }{ \signThickness } \par\smallskip } }


% Redefinindo as cores dos links (pacote hyperref)
% O hyperref é incluso automaticamente pelo abntex2
\hypersetup{pageanchor=false,
	colorlinks=true,
	linkcolor=blue,
	citecolor=blue,
	urlcolor=blue,
	linktocpage=true}


% Redefinicao de alguns comandos do pacote algorithm2e
\SetKwBlock{Inicio}{in\'{i}cio}{fim}
\SetKwFor{Para}{para}{fa\c{c}a}{fim para}%
\SetKwFor{ParaCada}{para cada}{fa\c{c}a}{fim para}%
\SetKwIF{Se}{SenaoSe}{Senao}{se}{ent\~{a}o}{sen\~{a}o se}{sen\~{a}o}{fim se}%
\SetKwRepeat{Repita}{repita}{at\'{e}}%


% Recuo do parágrafo em 1.5cm
\setlength{\parindent}{1.5cm}
%\OnehalfSpacing
\doublespacing


%Redefinição de espaços antes de subseções, seções e capítulos
\makeatletter
\newskip\beforesubsectionskip
\setlength{\beforesubsectionskip}{0.5em}
\newskip\beforesectionskip
\setlength{\beforesectionskip}{0.5em}
\newskip\beforechapterskip
\setlength{\beforechapterskip}{1.5em}
%\setlength{\cftbeforesubsectionskip}{0.5em}
%\setlength{\cftbeforesectionskip}{0.5em}
%\setlength{\cftbeforechapterskip}{1.5em}
\makeatother


%Definindo environment agradecimentos
\newenvironment{agradecimentos}[1][\agradecimentosnome]{%
	\begin{center}
		{\Large \bfseries \agradecimentosnome}
	\end{center}
}


%Definindo environment epígrafe
\newenvironment{epigrafe}[1][\epigrafenome]{%
	\begin{center}
		{\Huge \bfseries \epigrafenome} %Comentar essas 3 linhas se não quiser que o nome Epígrafe ou Quote apareça no cabeçalho
	\end{center}
}


\newcommand{\imprimircapa}{%
	\begin{titlepage}
		\begin{center}
			
			% Cabeçalho (não deve ser modificado)
			% Contém o brasão da Universidade, o logotipo do Departamento, além dos dados
			% relacionados à vinculação do aluno (Universidade, Centro, Departamento e Curso)
			\begin{minipage}{2.5cm}
				\begin{center}
					\includegraphics[width=2.5cm]{imagens/logos/logo_UFRN.png}
				\end{center}
			\end{minipage}
			\begin{minipage}{11cm}
				\begin{center}
					\begin{singlespace}
						{\small \textsc{Universidade Federal do Rio Grande do Norte}
							\\
							\textsc{Centro de Ciências Exatas e da Terra}
							\\
							\textsc{Departamento de Informática e Matemática Aplicada}
							\\
							\textsc{Programa de Pós-Graduação em Sistemas e Computação}
							\\
							\iftoggle{PPgSC-Tese}
								{\textsc{Doutorado Acadêmico em Ciência da Computação}}
								{\textsc{Mestrado Acadêmico em Sistemas e Computação}}
						}
					\end{singlespace}
				\end{center}
			\end{minipage}
			\begin{minipage}{1.8cm}
				\begin{center}
					\includegraphics[width=1.8cm, height=2cm]{imagens/logos/logo-ppgsc.png}
				\end{center}
			\end{minipage}
			
			\vspace{5cm}
			
			% Título do trabalho
			{\setlength{\baselineskip}
				{1.3\baselineskip}
				{\LARGE \textbf{\Titulo}}\par}
			
			\vspace{4cm}
			
			% Nome do aluno (autor)
			
			{\large \textbf{\Autor}}
			
			\vspace{6cm}
			
			% Local da instituição onde o trabalho deve ser apresentado e ano de entrega do mesmo
			\Local\\\Data 
		\end{center}
	\end{titlepage}
	
	% Solução para geração de páginas duplicadas, uma delas fica em branco
	\hypersetup{pageanchor=true}
}


% Conteudo padrao da Folha de Rosto
\makeatletter

\newcommand{\folhaderosto} {
	\begin{center}
		
		{\bfseries \Large \Autor{}}
		
		\vspace*{\fill}\vspace*{\fill}
		\begin{center}
			\bfseries \Large \Titulo
		\end{center}
		\vspace*{\fill}
		
		\hspace{.45\textwidth}
		\begin{minipage}{.5\textwidth}
			\singlespacing
			\preambulo
		\end{minipage}%
		\vspace*{\fill}
		
		{\large Orientador~\par\Orientador\par}
%		\if\boolean{COorientador}{%
%			{\large Coorientador \\ \Coorientador}
%		}%
		\iftoggle{CO-orientador}{%
		{\large Coorientador \\ \Coorientador}
		}%
	    {}
		\vspace*{\fill}
		
		\textsc{\Instituicao}\vspace*{\fill}
		
		{\large\Local}
		\par
		{\large\Data}
		\vspace*{1cm}
		
	\end{center}
	\newpage
	}
	
\makeatother


%Definindo environment resumo para classe scrbook (não faz parte da classe original)
\makeatletter
\newenvironment{resumo}{%
	\if@titlepage
	\titlepage
	\null\vfil
	\@beginparpenalty\@lowpenalty
	\begin{center}
		{\Large{\textbf{\Titulo}}}
	\end{center}
	
	\vspace{0.3cm}
	
	\begin{flushright}
		Autor:~\Autor\\
		Orientador:~\Orientador\\
		\iftoggle{CO-orientador}{Coorientador:~\Coorientador}{}
	\end{flushright}
	
	\begin{center}%
		\Huge \bfseries Resumo
		\@endparpenalty\@M
	\end{center}
	\quotation
	\fi}%
{\par\vfil\null\endtitlepage}
\makeatother


%Definindo environment abstract para classe scrbook (não faz parte da classe original)
\makeatletter
\newenvironment{abstract}{%
	\if@titlepage
	\titlepage
	\null\vfil
	\@beginparpenalty\@lowpenalty
	\begin{center}
		{\Large{\textbf{\TituloEstrangeiro}}}
	\end{center}
	
	\vspace{0.3cm}
	
	\begin{flushright}
		Author:~\Autor \\
		Advisor:~\Orientador \\
		\iftoggle{CO-orientador}{Co-advisor:~\Coorientador}{}
	\end{flushright}
	
	\begin{center}%
		\Huge \bfseries Abstract
		\@endparpenalty\@M
	\end{center}
	\quotation
	\fi}%
{\par\vfil\null\endtitlepage}
\makeatother

%Definindo o comando BibTeX para imprimir o logo. Baixei da Internet.
\def\BibTeX{\textrm{B\kern-.05em\textsc{i\kern-.025em b}\kern-.08em T\kern-.1667em\lower.7ex\hbox{E}\kern-.125emX}}