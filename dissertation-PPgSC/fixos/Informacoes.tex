\newcommand{\Local}{Natal-RN}

\newcommand{\Curso}{Ciência da Computação}

\newcommand{\Instituicao}{%
  PPgSC -- Programa de Pós-Graduação em Sistemas e Computação\par 
  DIMAp -- Departamento de Informática e Matemática Aplicada\par
  CCET -- Centro de Ciências Exatas e da Terra\par
  UFRN -- Universidade Federal do Rio Grande do Norte
}

\newcommand{\TipoTrabalho}{Trabalho de Conclusão de Curso}
\iftoggle{PPgSC-Tese}
{\iftoggle{PPgSC-Proposta}
	{\newcommand{\preambulo}{Proposta de Doutorado apresentada ao Programa de Pós-Graduação em Sistemas e Computação do Departamento de Informática e Matemática Aplicada da Universidade Federal do Rio Grande do Norte como requisito parcial para a obtenção do grau de Doutor em Ciência da Computação.\bigskip
		
    \textit{Linha de pesquisa}: 

    \LinhaDePesquisa{}}}
	{\newcommand{\preambulo}{Tese de Doutorado apresentada ao Programa de Pós-Graduação em Sistemas e Computação do Departamento de Informática e Matemática Aplicada da Universidade Federal do Rio Grande do Norte como requisito parcial para a obtenção do grau de Doutor em Ciência da Computação.\bigskip
		
	\textit{Linha de pesquisa}: 
		
	\LinhaDePesquisa{}}}
}
{\iftoggle{PPgSC-Proposta}
	{\newcommand{\preambulo}{Qualificação de Mestrado apresentada ao Programa de Pós-Graduação em Sistemas e Computação do Departamento de Informática e Matemática Aplicada da Universidade Federal do Rio Grande do Norte como requisito parcial para a obtenção do grau de Mestre em Sistemas e Computação.\bigskip
		
    \textit{Linha de pesquisa}: 
		
    \LinhaDePesquisa{}}}
	{\newcommand{\preambulo}{Dissertação de Mestrado apresentada ao Programa de Pós-Graduação em Sistemas e Computação do Departamento de Informática e Matemática Aplicada da Universidade Federal do Rio Grande do Norte como requisito parcial para a obtenção do grau de Mestre em Sistemas e Computação.\bigskip

	\textit{Linha de pesquisa}: 

	\LinhaDePesquisa{}}}
}

