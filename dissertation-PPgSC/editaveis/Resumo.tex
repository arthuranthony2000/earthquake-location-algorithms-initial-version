% Resumo em língua vernácula
\begin{resumo}
Este documento descreve o modelo de dissertações e teses do \gls{ppgsc} da \gls{ufrn}, sua organização e as razões que embasaram algumas escolhas. O principal objetivo deste documento é facilitar a escrita dos documentos por parte dos alunos, e homogeneizar a diagramação dos mesmos. Ele não se propõe a ser um guia inicial para quem não conhece \LaTeX{}, mas indica referências que podem ser utilizadas tanto para usuários novatos como para os que possuem larga experiência em diagramação usando \LaTeX{}. A maioria dos pacotes incluídos neste modelo são brevemente descritos e alguns exemplos são incluídos juntamente com os códigos \LaTeX{} que os produzem. Este modelo ficará disponível com os arquivos fonte, assim os usuários terão acesso ao código \LaTeX{} usado para gerar todos os exemplos, como figuras, tabelas, algoritmos e códigos. Sugiro que você identifique os pacotes que você deve utilizar e comente a inclusão do restante, de modo a diminuir o tempo de processamento do documento, e inclua-os a medida que os necessite.
 
  \bigbreak

  \noindent
  \textit{Palavras-chave}: \LaTeX{}, \gls{ppgsc}, \gls{ufrn}.
\end{resumo}