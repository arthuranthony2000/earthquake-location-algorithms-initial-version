\chapter{Apêndice 1}

Um apêndice deve conter material complementar a sua dissertação/tese, que serve para complementar a argumentação do autor sobre seu trabalho. O conteúdo de um apêndice deve ter sido elaborado pelo autor. Caso o autor deseje incluir conteúdo complementar que auxilie sua argumentação, mas que tenha sido elaborado por outras pessoas, o mesmo deve adicionar esse conteúdo em um anexo.

Os apêndices dever ser localizados após as referências, e os anexos após os apêndices, caso existam. Ambos devem aparecer antes do índice remissivo. É desejável que ambos apareçam no sumário. Para incluir apêndices, você deve ter o pacote \texttt{appendix} \parencite{appendix} instalado, cujo manual está disponível em \url{https://ctan.dcc.uchile.cl/macros/latex/contrib/appendix/appendix.pdf}.

Caso não necessite de um apêndice, comente os comandos abaixo no arquivo \texttt{DissertacaoPPgSC.tex}.

\begin{listing}[ht]
	\begin{minted}[ linenos=true, autogobble, bgcolor=Cornsilk1 ]{tex}
		\begin{appendices}
		  % Apêndice A (arquivo capitulos/apendice1.tex)
		  \chapter{Apêndice 1}

Um apêndice deve conter material complementar a sua dissertação/tese, que serve para complementar a argumentação do autor sobre seu trabalho. O conteúdo de um apêndice deve ter sido elaborado pelo autor. Caso o autor deseje incluir conteúdo complementar que auxilie sua argumentação, mas que tenha sido elaborado por outras pessoas, o mesmo deve adicionar esse conteúdo em um anexo.

Os apêndices dever ser localizados após as referências, e os anexos após os apêndices, caso existam. Ambos devem aparecer antes do índice remissivo. É desejável que ambos apareçam no sumário. Para incluir apêndices, você deve ter o pacote \texttt{appendix} \parencite{appendix} instalado, cujo manual está disponível em \url{https://ctan.dcc.uchile.cl/macros/latex/contrib/appendix/appendix.pdf}.

Caso não necessite de um apêndice, comente os comandos abaixo no arquivo \texttt{DissertacaoPPgSC.tex}.

\begin{listing}[ht]
	\begin{minted}[ linenos=true, autogobble, bgcolor=Cornsilk1 ]{tex}
		\begin{appendices}
		  % Apêndice A (arquivo capitulos/apendice1.tex)
		  \chapter{Apêndice 1}

Um apêndice deve conter material complementar a sua dissertação/tese, que serve para complementar a argumentação do autor sobre seu trabalho. O conteúdo de um apêndice deve ter sido elaborado pelo autor. Caso o autor deseje incluir conteúdo complementar que auxilie sua argumentação, mas que tenha sido elaborado por outras pessoas, o mesmo deve adicionar esse conteúdo em um anexo.

Os apêndices dever ser localizados após as referências, e os anexos após os apêndices, caso existam. Ambos devem aparecer antes do índice remissivo. É desejável que ambos apareçam no sumário. Para incluir apêndices, você deve ter o pacote \texttt{appendix} \parencite{appendix} instalado, cujo manual está disponível em \url{https://ctan.dcc.uchile.cl/macros/latex/contrib/appendix/appendix.pdf}.

Caso não necessite de um apêndice, comente os comandos abaixo no arquivo \texttt{DissertacaoPPgSC.tex}.

\begin{listing}[ht]
	\begin{minted}[ linenos=true, autogobble, bgcolor=Cornsilk1 ]{tex}
		\begin{appendices}
		  % Apêndice A (arquivo capitulos/apendice1.tex)
		  \chapter{Apêndice 1}

Um apêndice deve conter material complementar a sua dissertação/tese, que serve para complementar a argumentação do autor sobre seu trabalho. O conteúdo de um apêndice deve ter sido elaborado pelo autor. Caso o autor deseje incluir conteúdo complementar que auxilie sua argumentação, mas que tenha sido elaborado por outras pessoas, o mesmo deve adicionar esse conteúdo em um anexo.

Os apêndices dever ser localizados após as referências, e os anexos após os apêndices, caso existam. Ambos devem aparecer antes do índice remissivo. É desejável que ambos apareçam no sumário. Para incluir apêndices, você deve ter o pacote \texttt{appendix} \parencite{appendix} instalado, cujo manual está disponível em \url{https://ctan.dcc.uchile.cl/macros/latex/contrib/appendix/appendix.pdf}.

Caso não necessite de um apêndice, comente os comandos abaixo no arquivo \texttt{DissertacaoPPgSC.tex}.

\begin{listing}[ht]
	\begin{minted}[ linenos=true, autogobble, bgcolor=Cornsilk1 ]{tex}
		\begin{appendices}
		  % Apêndice A (arquivo capitulos/apendice1.tex)
		  \input{capitulos/apendice1.tex}
		\end{appendices}
	\end{minted}
	\caption{Exemplo de código \LaTeX{} usado para carregar um arquivo de apêndice.}
	\label{cod:appendix}
\end{listing}
		\end{appendices}
	\end{minted}
	\caption{Exemplo de código \LaTeX{} usado para carregar um arquivo de apêndice.}
	\label{cod:appendix}
\end{listing}
		\end{appendices}
	\end{minted}
	\caption{Exemplo de código \LaTeX{} usado para carregar um arquivo de apêndice.}
	\label{cod:appendix}
\end{listing}
		\end{appendices}
	\end{minted}
	\caption{Exemplo de código \LaTeX{} usado para carregar um arquivo de apêndice.}
	\label{cod:appendix}
\end{listing}